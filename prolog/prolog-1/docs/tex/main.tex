\section{Задание} с
Cоставить программу - базу знаний, с помощью которой можно определить, например, множество студентов, обучающихся в одном ВУЗе и их телефоны. 
Студент может одновременно обучаться в нескольких ВУЗах. Привести примеры возможных вариантов вопросов и варианты ответов (не менее 3-х). 
Описать порядок формирования варинатов ответа.

\section{Теоретические вопросы}

\subsection{Что собой представляет програма на Prolog}

Программа на Prolog не является последовательностью действий, - она представляет собой набор фактов и 
правил, которые формируют базу знаний о предметной области. Факты представляют собой составные термы, 
с помощью которых фиксируется наличие истинностных отношений между объектами предметной области — аргументами 
терма. Правила являются обобщенной формулировкой условия истинности знания – отношения между объектами предметной 
области (аргументами терма), которое записано в заголовке правила. Условие истинности этого отношения является телом 
правила. Заголовок правила отделяется от тела правила символом «:-» , правило завершается символом « . ».

\subsection{Из чего состоит програма на Prolog}

Программа на Prolog состоит из разделов. Каждый раздел начинается со своего заголовка. Структура программы:

\begin{enumerate}
	\item директивы компилятора — зарезервированные символьные константы
	\item CONSTANTS — раздел описания констант
	\item DOMAINS — раздел описания доменов
	\item DATABASE — раздел описания предикатов внутренней базы данных
	\item PREDICATES — раздел описания предикатов
	\item CLAUSES — раздел описания предложений базы знаний
	\item GOAL — раздел описания внутренней цели (вопроса).
\end{enumerate}

В программе не обязательно должны быть все разделы.
